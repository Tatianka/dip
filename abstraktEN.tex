\null
We study modern regular expressions specified in programming language Python as formal model in theory of formal languages. New constructions extending basic regular languages are backreference, positive and negative lookaround. There are already known results about backreference and positive lookaround. We fill in the hierarchy of classes of regexes over different sets of operations with new results, e.g. incomparability with context-free languages. We show that closure under concatenation of class of regexes with positive lookaround is not trivial. Then we prove properties of negative lookaround similar to results about the positive version. In the field of space complexity we show that regex with positive lookaround needs $NSPACE(\log n)$ and with added negative lookaround there is needed $DSPACE(\log^2 n)$. If we get on input together regex and word, for class with positive lookaround we show space complexity $NSPACE(n\log n)$ and result $DSPACE(n\log^2 n)$ holds only if we can assure that there is only constant depth of nested lookarounds. For proving last few results we defined new formal model with configurations and step of computation inspired by Turing machine.