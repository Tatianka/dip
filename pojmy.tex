\chapter*{Použité pojmy a skratky}
\label{chap:pojmy}
\phantomsection
\addcontentsline{toc}{chapter}{Použité pojmy a skratky}

\begin{description}
\item[$\varepsilon$] -- prázdne slovo
\item[$\N$, $\N_0$] -- prirodzené čísla (bez nuly, s nulou)
\item[$ L_{1}L_{2} $] -- zreťazenie jazykov $ L_{1} $ a $ L_{2} $
\item[$ L^* $] -- Kleeneho iterácia ($L^*=\cup^{\infty}_{i=0}L^i$, kde $L^0=\lbrace \varepsilon \rbrace$, $L^1=L$ a $L^{i+1}=L^iL$)
\item[$ \R $] -- trieda regulárnych jazykov
\item[$ \L_{CF}$] -- trieda bezkontextových jazykov
\item[$ \L_{CS}$] -- trieda kontextových jazykov
\item[DKA/NKA] -- deterministický/nedeterministický konečný automat
\item[LBA] -- lineárne ohraničený Turingov stroj
\item[TS] -- Turingov stroj
\item[matchovať] -- keď regex $\alpha$ matchuje slovo $w$ (vyhlási zhodu), znamená to, že $w\in L(\alpha)$
\item[lookaround] -- spoločný názov pre lookahead a lookbehind

\item[$\re$] -- regexy nad množinou operácií, ktorými vieme popísať iba regulárne jazyky (základná definícia)
\item[$\e$] -- $\re$ so spätnými referenciami
\item[$\le$] -- $\e$ s operáciami lookahead a lookbehind
\item[$\nle$] -- $\le$ s operáciami negatívny lookahead a negatívny lookbehind
\item[$\rel$] -- trieda jazykov nad $\re$, ekvivalentná $\R$
\item[$\el$] -- trieda jazykov nad $\e$
\item[$\lel$] -- trieda jazykov nad $\le$
\item[$\nlel$] -- trieda jazykov nad $\nle$
\end{description}