$\\$
Skúmali sme moderné regulárne výrazy špecifikované v jazyku Python z hľadiska teórie formálnych jazykov. Nové konštrukcie rozširujúce klasické regulárne výrazy sú spätné referencie, pozitívny a negatívny lookaround. Práca naväzuje na výsledky o~spätných referenciách a pozitívnom lookarounde a doplňuje výsledky do hierarchie tried regexov nad rôznymi množinami operácií, ako napríklad neporovnateľnosť s bezkontextovými jazykmi. Ďalej ukazujeme, že uzavretosť triedy s pozitívnym lookaroundom na zreťazenie nie je triviálne a dokázali sme vlastnosti negatívneho lookaroundu podobné ako pri pozitívnej verzii. V~oblasti priestorovej zložitosti sme zistili, že regexy s~pozitívnym lookaroundom potrebujú priestor $NSPACE(\log n)$ a s pridaným negatívnym lookaroundom dosiahneme $DSPACE(\log^2 n)$. Pokiaľ na vstup dostávame zároveň regex aj slovo, zložitosť pre triedu s pozitívnym lookaroundom je $NSPACE(n\log n)$ a výsledok $DSPACE(n\log^2 n)$ platí iba ak je zaručená konštantná hĺbka vnorenia lookaroundov. K dôkazom sme zaviedli nový formálny model s~konfiguráciami a krokom výpočtu inšpirovaný Turingovým strojom. 