$\\$
Skúmali sme moderné regulárne výrazy špecifikované v jazyku Python z hľadiska teórie formálnych jazykov. Nové konštrukcie rozširujúce klasické regulárne výrazy sú spätné referencie, pozitívny a negatívny lookaround. Práca naväzuje na výsledky o spätných referenciách a pozitívnom lookarounde a doplňuje výsledky do hierarchie tried regexov, ako je neporovnateľnosť s bezkontextovými jazykmi. Zaoberá sa vlastnosťami negatívneho lookaroundu a ukazuje, že jeho pozitívna verzia ohrozuje uzavretosť na zreťazenie. V oblasti priestorovej zložitosti sme ukázali, že regexy s~pozitívnym lookaroundom potrebujú priestor $NSPACE(\log n)$ a s pridaným negatívnym lookaroundom treba $DSPACE(\log^2 n)$. Pokiaľ na vstup dostávame aj regex, zložitosť pre $\lel$ je $NSPACE(n\log n)$ a $DSPACE(n\log^2 n)$, ak je zaručená konštantná hĺbka vnorenia lookaroundov. K dôkazom sme zaviedli nový formálny model s konfiguráciami krokom výpočtu, inšpirovaný Turingovým strojom. 