\chapter[Súčasný stav problematiky]{Úvod do súčasneho stavu problematiky}
\label{chap:uvod}

Myšlienka regulárnych výrazov bola prvýkrát spomenutá v teórii formálnych jazykov a automatov ako iný spôsob popisu regulárnych jazykov. Vtedy pozostávali z operácií zjednotenia, zreťazenia a Kleeneho uzáveru. Z takéhoto zápisu bolo ľahšie vidieť, o aký jazyk ide, než z konečných automatov alebo regulárnych gramatík. Vďaka jednoduchosti boli implementované ako nástroj na vyhľadávanie slov zo špecifikovaného jazyka. Postupom času a s inšpiráciou zo strany užívateľov k nim pribúdali ďalšie operácie. Niektoré boli len skratkou k tomu, čo sa už dalo zapísať -- umožnili zapísať to isté, ale menej znakmi -- ostatné otvárali dvere k popisu dovtedy nedosiahnuteľných jazykov.

Zmes týchto operácií nazývame moderné regulárne výrazy alebo regexy. Tento model opäť vraciame do teórie jazykov a skúmame jeho vlastnosti, zaradenie v Chomského hierarchii a zložitosť.

\section{Základné definície}
\label{definicie}

Každý regulárny výraz sa skladá zo znakov a metaznakov. Znak je regulárny výraz reprezentujúci sám seba, teda $L(a) = \lbrace a \rbrace$. Metaznaky popisujú operáciu nad regulárnymi výrazmi. Ak potrebujeme použiť metaznak ako znak, stačí pred neho dať $\backslash$, teda $L(\backslash x) = \lbrace x \rbrace$. Ak metaznak vyžaduje vstup, je ním posledný znak/metaznak/uzátvorkovaný výraz naľavo od neho.

Nech $\alpha,\beta$ sú regexy. Základné operácie sú:
\begin{itemize}
\item zreťazenie $(\alpha)(\beta)$
\item alternácia $(\alpha)|(\beta)$
\item Kleeneho uzáver $(\alpha)*$
\end{itemize}
Platí $L((\alpha)(\beta)) = L(\alpha)L(\beta)$, $L((\alpha)|(\beta)) = L(\alpha)\cup L(\beta)$ a $L((\alpha)*) = L(\alpha)^*$.

Aby sme špecifikovali pole pôsobnosti operácií, používame v regulárnych výrazoch okrúhle zátvorky (). Je možné ich vynechať, potom sa operácie budú vyhodnocovať podľa priorít -- tie uvedieme po definovaní všetkých operácií.

Regulárny výraz je vykonávaný zľava doprava. Keď sa dostaneme na koniec výrazu aj slova (za posledný znak), hovoríme, že výraz matchuje\footnote{Preklad ,,regex sa zhoduje so slovom'' nevyjadruje presne to, čo je vyjadrené v angličtine. Znie to, akoby regex bol to isté, čo slovo. Preto sme sa rozhodli zostať pri slove match a slovese matchovať.} (z angl. \textit{match} -- zhoda) slovo, výraz vyhovuje slovu, alebo slovo vyhovuje výrazu.

\subsection{Nové konštrukcie}

Najprv si uvedieme konštrukcie, ktoré sú len ,,kozmetickou úpravou"" základných regulárnych výrazov. Všetky sa dajú zapísať pomocou základných operácií, ale nový zápis je stručnejší a prehľadnejší. Pre formálnejšiu definíciu odporúčame siahnuť po \cite{mojaBak}.
\begin{itemize}
\item $+$ -- opakuj 1 alebo viackrát: $(\alpha)+=\alpha(\alpha)*$
\item $\lbrace \rbrace$ -- zložené zátvorky sú používané ako $\{n,m\}$ (opakuj aspoň $n$ a najviac $m$-krát) a $\{n\}=\{n,n\}$ (opakuj $n$-krát)
\item $[~]$ -- predstavuje ľubovoľný znak, z tých, ktoré má vnútri zapísané. Vieme použiť aj intervaly, napr. a--z, A--Z, 0--9, \dots~a kombinovať ich. Všetky metaznaky vnútri [~] sa považujú za normálne znaky.
\item $[\textasciicircum~]$ -- predstavuje ľubovoľný znak, ktorý nepatrí medzi zapísané. Ostatné ako [~].
\item ? -- ak samostatne: opakuj 0 alebo 1-krát \\
ak za operáciou: namiesto greedy implementácie použi minimalistickú, t.j. zober čo najmenej znakov (platí pre $*,+,?,\lbrace n,m \rbrace$)\footnote{Všetky spomenuté operácie ,,žerú'' znaky a na ich implementáciu bol použitý greedy algoritmus.}
\item . -- predstavuje ľubovoľný znak
\item $\textasciicircum$ -- metaznak označujúci začiatok slova
\item \$ -- metaznak označujúci koniec slova
\end{itemize}

Okrem operácií označených metaznakmi vznikli aj zložitejšie operácie, na popis ktorých treba dlhšie konštrukcie. V prvom rade sa zaviedlo \textbf{číslovanie okrúhlych zátvoriek}. Čísluje sa zľava doprava podľa poradia ľavej (otváracej) zátvorky, ale konštrukcie tvaru (?\dots) sa vynechávajú. Toto číslovanie sa deje automaticky pri každom matchovaní, teda už pri písaní výrazu ho môžeme využívať. Popíšme si teraz zložitejšie operácie:

\begin{itemize}
\item \textbf{komentár} ozn. (?\#text komentáru) -- klasický komentár, určený čitateľom kódu; nemá vplyv na výraz, pri matchovaní sa ignoruje
\item \textbf{spätné referencie} ozn. $\backslash k$, $k \in\N$ -- môže sa nachádzať na ľubovoľnom mieste vo výraze ZA $k$-tou pravou (zatváracou) zátvorkou. Odkazuje sa na $k$-te zátvorky a presný význam sa určuje až pri hľadaní zhody na konkrétnom vstupe. Algoritmus si zapamätá aké podslovo zo vstupu matchoval výraz vnútri $k$-tych zátvoriek a presne toto podslovo matchuje konštrukcia , keď ďalej vo výraze narazí na $\backslash k$. Počas matchovania môžu $k$-te zátvorky matchovať viackrát. V momente, keď ideme vykonávať $\backslash k$ si vyberáme doteraz posledné z nich.

Ukážeme si to na príklade, ako regex $\displaystyle \alpha\mathop{(}_k\beta\mathop{)}_k\gamma\backslash k\delta$ matchuje slovo $w$:
$$w = \underbrace{x_1\dots x_{i-1}}_\alpha 
 \overbrace{\underbrace{x_i\dots x_{j-1}}_{ \displaystyle{\mathop(_k\beta \mathop)_k}}}^{w_k} 
 \underbrace{x_j\dots x_{l-1}}_\gamma 
 \overbrace{\underbrace{x_l\dots x_{m-1}}_{\backslash k}}^{w_k}
 \underbrace{x_{m}\dots x_{n}}_\delta$$
musí platiť $w_k= x_i\dots x_{j-1} = x_l\dots x_{m-1} $ a zároveň $w_k$ je posledné matchované podslovo $k$-tych zátvoriek v momente, keď sme narazili na konštrukciu $\backslash k$.


Predpokladáme, že číslo $k$ je zapísané znakmi z inej abecedy ako zvyšok regexu -- t.j. je jednoznačne určiteľné, kde končí zápis $k$.\footnote{V implementovaných regexoch vieme deterministicky určiť, kde končí číslo $k$ -- sú povolené maximálne trojciferné čísla. My chceme všeobecnejší model, bez obmedzení na veľkosť $k$, a práve preto je tento predpoklad oprávnený. Zároveň si to môžeme dovoliť, lebo ak máme $k$ zapísané ako číslo v desiatkovej sústave, v teórii je jednoduché zasubstituovať hľadaný jazyk tak, aby neobsahoval znaky 0,\dots,9. Viac informácií o obmedzeniach regexov v praxi nájdete v \cite[kapitola Prax]{mojaBak}.}

\item \textbf{lookahead} -- nazeranie dopredu.
\begin{itemize}
\item \textbf{pozitívny} ozn. $\lookahead\alpha)$, kde $\alpha$ je nejaký moderný regulárny výraz 

V momente, keď na lookahead narazíme, zapamätáme si aktuálnu pozíciu v slove. Od tejto pozície začneme hľadať zhodu s výrazom $\alpha$. Akonáhle vyhlási zhodu (nemusí ani dočítať slovo do konca), vrátime sa naspäť na zapamätané miesto a odtiaľ pokračujeme v slove vykonávaním regexu za lookaheadom -- akokeby tam lookahead nikdy nebol. Inak povedané lookahead nevyžiera písmenká a robí akýsi prienik. Ak nenájde zhodu na žiadnom prefixe od zapamätaného miesta, znamená to nezhodu pre celý regex -- ak sa dá, musíme sa vo výpočte vrátiť naspäť a vyskúšať inú možnosť predošlých operácií.\footnote{V praxi je lookahead atomická operácia, ale my opäť chceme model v plnej sile.} Pokiaľ chceme, aby lookahead slovo dočítal do konca, použijeme metaznak \$.

Ukážme si priebeh matchovania regexom $\alpha\lookahead\beta)\gamma$ na slove $w$. Vidíme, že $\beta$ a $\gamma$ začínajú na rovnakom mieste v slove.
 $$w = \underbrace{x_1\dots x_{i-1}}_\alpha \underbrace{\overbrace{x_i \dots x_j}^\beta x_{j+1} \dots x_n }_\gamma$$ 
\item \textbf{negatívny} ozn. $\nlookahead\alpha)$, kde $\alpha$ je nejaký moderný regulárny výraz  

Nesmie nájsť slovo z jazyka $L(\alpha)$. Postup je ako pri lookaheade, ale končí úspešne len ak by lookahead zamietol.
\end{itemize}
\item \textbf{lookbehind} -- nazeranie dozadu.
\begin{itemize}
\item \textbf{pozitívny} ozn. $\lookbehind\alpha)$, kde $\alpha$ je nejaký moderný regulárny výraz 

Hľadá slovo z $L(\alpha)$ naľavo od aktuálneho miesta v slove (musí končiť na susediacom políčku vľavo od aktuálnej pozície v slove). Opäť nie je určená hranica slova, môže začínať kdekoľvek, ak nie je vynútený začiatok slova znakom $\textasciicircum$. 

Ak by sme chceli deterministický algoritmus, vyzeral by nasledovne: od teraz do konca vykonávania lookbehindu bude na pozícii, na ktorej v slove sme, špeciálny znak pre koniec slova -- endmarker (nie \$). Najprv vyskúšame 1 susedný symbol naľavo, či patrí do jazyka. Ak nie skúsime čítať o jeden znak viac (2 symboly naľavo), keď neuspejeme, opäť posunieme pomyselný začiatok slova doľava. Ak akceptujeme, môže to byť len na endmarkeri. Ak sme neakceptovali a začiatok slova nejde viac posunúť, zamietneme.

Opäť to bude dobre vidieť na príklade, výpočet regexu $\alpha\lookbehind\beta)\gamma$ na slove $w$. Podslová $\alpha,\beta$ končia na tej istej pozícii.

$$w = \underbrace{x_1\dots x_{i-1} \overbrace{x_i \dots x_j}^\beta}_\alpha \underbrace{x_{j+1} \dots x_n }_\gamma$$
\item \textbf{negatívny} ozn. $\nlookbehind\alpha)$, kde $\alpha$ je nejaký moderný regulárny výraz 

Hľadá ako pozivíny lookbehind, ale akceptuje len ak neexistuje slovo z~jazyka $L(\alpha)$ vyskytujúce sa naľavo od aktuálnej pozície.
\end{itemize}
\end{itemize}

Pre posledné 2 operácie zostaneme pri ich anglických názvoch, pretože sú stručnejšie a zvučnejšie. Existuje pre ne aj spoločný názov \underline{lookaround}, takisto s prívlastkom pozitívny/negatívny, ak myslíme iba ich pozitívne/negatívne verzie.

Formálnejšiu definíciu zložitejších konštrukcií nájdete v \cite{mojaBak}.
\\ \par
Moderné regulárne výrazy sa skladajú z \textbf{konečného počtu} znakov, metaznakov a zložitejších operácií.

\subsubsection{Priotity operácií}

Ako sme už spomínali, pre korektné matchovanie regexom potrebujeme vedieť priority operácií. Zároveň nám to umožní niekedy vynechať zátvorky a regex tak bude prehľadnejší.

Nové konštrukcie lookahead a lookbehind neberú zo slova žiadne písmenká. Môžeme si ich predstaviť ako overenie nejakej podmienky, je to taká odbočka od výpočtu. Preto nemá zmysel na ne aplikovať operácie ako $*,+,?,\lbrace\rbrace,|$ a teda taký regex nebude validný. Je možné ich uzavrieť do ďalších lookaroundov, je možné sa odkazovať na zátvorky, ktoré majú vnútri (iba ak sa jedná o pozitívny lookaround, odkazovať sa dovnútra negatívnejho lookaroundu nemá zmysel). Pokiaľ však aplikujeme operáciu na nejaký regex obsahujúci lookaround, prirodzene je lookaround vykonaný zakaždým, keď naň príde rad.

Niektoré operácie sa správajú ako znak -- $\left[~\right]$, [\textasciicircum~] , ., \textasciicircum, \$, $\backslash k$ -- teda je s nimi možné narábať ako s obyčajným znakom (okrem \textasciicircum a \$, tieto nemá zmysel iterovať (napr. $\$*$) -- existuje vždy práve 1 začiatok a 1 koniec slova). Ostatné kombinujú znaky s nasledovnými prioritami:
\begin{center}
\begin{tabular}{|c||c|c|c|c|}
\hline
priorita &najvyššia&<&<& najnižšia  \\
\hline
operácia & () & $*$ + ? $\lbrace \rbrace$ & zreťazenie & | \\ \hline
\end{tabular}
\end{center}

Máme veľkú množinu operácií a budeme chcieť rozlišovať regexy používajúce rôzne operácie. Preto si zavedieme triedy regexov nad konkrétnymi množinami operácií. Regex z takejto triedy bude obsahovať iba operácie z povolenej množiny, v ľubovoľnom počte (stále však platí, že regex je konečnej dĺžky).

Zároveň chceme pomenovať triedy jazykov nad jednotlivými triedami regexov.

\begin{description}
\item[$\re$] -- trieda regexov nad množinou operácií, pomocou ktorých vieme popísať iba regulárne jazyky. Množina operácií obsahuje všetky znaky a metaznaky bez zložitejších operácií.
\item[$\e$] -- $\re$ so spätnými referenciami
\item[$\le$] -- $\e$ s pozitívnym lookaroundom
\item[$\nle$] -- $\le$ s negatívnym lookaroundom
\item[$\rel$] -- trieda jazykov nad $\re$ ($= \R$)
\item[$\el$] -- trieda jazykov nad $\e$
\item[$\lel$] -- trieda jazykov nad $\le$
\item[$\nlel$] -- trieda jazykov nad $\nle$
\end{description}

\section[Vlastnosti a sila]{Vlastnosti a sila moderných regulárnych výrazov}
\label{usila}

Keď sme hľadali články o moderných regulárnych výrazoch, našli sme výsledky prevažne o regulárnych výrazoch schopných generovať iba regulárne jazyky a preštudovaná bola trieda $\e$ \cite{ExtendedRegexPower}, \cite{ExtendedRegexIntersec} (v článkoch pomenovaná EREG). Tému moderných regulárnych výrazov sme rozvinuli v bakalárskej práci \cite{mojaBak}, v tejto práci naviažeme na tieto výsledky. Popíšme si teraz poznatky nadobudnuté v oblasti nášho záujmu.

Keďže operácie lookahead a lookbehind sú v regexoch nové, na počiatku sme skúmali ich vlastnosti. Uzáverové vlastnosti tried Chomského hierarchie dopadli nasledovne:
\begin{veta} \textbf{\emph{\cite[Veta 2.2.5.]{mojaBak}}}\label{lookahead+R} \ \par
Trieda $\R$ je uzavretá na pozitívny lookaround.
\end{veta}
Trieda $\L_{CS}$ je tiež uzavretá a trieda $\L_{CF}$ nie je.

Základnou vlastnosťou pozitívneho lookaheadu je, že do jeho vnútra sa oplatí dávať iba prefixové jazyky. Vyplýva to z toho, že lookahead matchuje tak skoro, ako len môže -- akonáhle nájde prvé slovo patriace do jeho jazyka, zvyšok vstupu už neskúma. A tým pádom môžeme z daného jazyka vyhodiť všetky slová, ktorých prefix doň tiež patrí. Analogicky to platí pre pozitívny lookbehind a sufixové jazyky. Dôsledkom je, že pokiaľ jazyk v pozitívnom lookarounde obsahuje prázdne slovo ($\varepsilon$), potom je možné celý lookaround vynechať. Dôvodom je to, že lookaround ,,nevyužiera písmenká'', preto si môže zakaždým zvoliť matchovať $\varepsilon$ a automaticky vždy akceptovať.

Prejdime teraz k regexom. Pridaním pozitívneho lookaroundu k triede $\re$ získame nový model. Avšak silnejší ako $\re$ nie je:
\begin{veta} \textbf{\emph{\cite[Veta 2.2.10.]{mojaBak}}}\label{lookaround+R} \ \par
Trieda jazykov nad $\re$ s pozitívnym lookaroundom je $\R$.
\end{veta}

Skúmaním modelu so spätnými referenciami aj pozitívnym lookaroundom vznikla hierarchia:
\begin{veta}\textbf{\emph{\cite[Vety 2.2.13 a 2.2.14.]{mojaBak}}}\label{le+lcs} \ \par
$ \el \subsetneq \lel \subseteq \L_{CS} $
\end{veta}

Z viet \ref{lookaround+R} a \ref{le+lcs} vyplýva dôležité pozorovanie, že síce lookaround samostatne regexom silu nepridáva, ale ak ho pridáme k modelu so spätnými referenciami, pridá do triedy nové jazyky. Z tohto hľadiska naberá na význame a vidíme dôvod, prečo sa ním zaoberať.

Je to spôsobené tým, že trieda $\el$ nie je uzavretá na prienik \cite{ExtendedRegexIntersec} a pozitívny lookaround je nástroj na zapísanie prieniku 2 jazykov. Navyše pridaním negatívneho lookaroundu je zaručená aj uzavretosť na komplement. Avšak to, či je naň uzavretá aj $\lel$ je otvorený problém.

Zaujímavá vlastnosť triedy $\el$ je, že obsahuje jednoduché jazyky a to v tom zmysle, že pre ňu existuje pumpovacia lema (\cite{ExtendedRegexPower} aj \cite{ExtendedRegexIntersec}). Pridaním look\-aroundu táto vlastnosť mizne -- do triedy $\lel$ patrí jazyk všetkých platných výpočtov Turingovho stroja \cite[Veta 2.2.16.]{mojaBak}. Pre unárne jazyky z $\lel$ bola dokázaná kvázi-pumpovacia lema. Kvázi preto, že platí len pre množinu regexov, ktoré vnútri Kleeneho $*$ neobsahujú žiadnu konštrukciu lookaheadu obsahujúceho metaznak \$ alebo lookbehindu obsahujúceho metaznak \textasciicircum ~(t.j. lookaround spúšťaný niekoľkokrát v iterácii, vždy potenciálne z inej pozície je problém).


\section[Popisná zložitosť]{Popisná zložitosť regulárnych výrazov}
\label{uzlozitost}

V cieľoch tejto práce je skúmať aj zložitosť moderných regulárnych výrazov. Opäť sa odvolávame na to, že k našej téme sme nenašli články. Je zrejmé, že skúmať NSPACE a DSPACE základných regulárnych výrazov nemá zmysel -- sú konštantné. Preto zostala popisná zložitosť.

Z oblasti popisnej zložitosti sú známe len výsledky pre regulárne výrazy s operáciami zjednotenia, zreťazenia a Kleeneho $*$ (RE), prípadne ešte prieniku a komplementu (GRE). Navyše mali ešte znak pre prázdny jazyk $\emptyset$ a prázdne slovo $\varepsilon$.

Spomeňme si najprv ako možno zadefinovať popisnú zložitosť \cite{newResults} \cite{compMeasures75}:

\begin{list}{$\bullet$}{Nech E je regulárny výraz, potom}
\item $|E|$ je jeho celková dĺžka
\item $rpn(E)$ je jeho celková dĺžka v reverznej poľskej notácii. Táto notácia oproti klasickej neobsahuje okrúhle zátvorky a používa explicitný metaznak pre zreťazenie~$\bullet$. Napríklad regulárny výraz $(a|ab)*(c|d)$ má 12 znakov a v reverznej poľskej notácii $aab\bullet |*cd|\bullet$ má 10 znakov.

Uvedomme si, že dĺžka reverznej poľkej notácie je rovnaká ako počet vrcholov v syntaktickom strome pre daný výraz. Preto možno vernejšie popisuje skutočnú zložitosť regulárneho výrazu (žiadne pomocné symboly, len znaky a operácie). Nakoľko však nie je veľmi prehľadná a ťažko sa v nej hľadajú chyby, nie je tak často používaná.
\item $|alph(E)|= N(E)$ je počet alfabetických symbolov v E
\item $H(E)$ je hĺbka vzhľadom na $*$, počet vnorení $*$
\item $L(E)$ je dĺžka najdlhšej neopakujúcej sa cesty cez výraz
\item $W(E)$ je šírka, počet zjednotených symbolov. Za touto mierou zložitosti nie je žiadna intuitívna predstava. Ako vidieť neskôr v definícii, dôvodom jej vzniku bola duálnosť k L.
\end{list}

V nasledujúcej tabuľke sú uvedené induktívne definície jednotlivých mier zložitosti. Zložitosť regulárneho jazyka vzhľadom na ľubovoľnú z týchto mier zložitosti je minimálna miera cez všetky regulárne výrazy pre daný regulárny jazyk.
\begin{center}

\begin{tabular}{|c||c|c|c|c|}
\hline
 ~ & Alphabetical Symbol & E $\cup$ F & E $\cdot$ F & E*
\\ \hline\hline
N & 1 & $N(E)+N(F)$ & $N(E)+N(F)$ & $N(E)$
\\ \hline 
H & 0 & $max(H(E),H(F))$ & $max(H(E),H(F))$ & $H(E)+1$
\\ \hline
L & 1 & $max(L(E),L(F))$ & $L(E)+L(F)$ & $L(E)$
\\ \hline
W & 1 & $W(E)+W(F)$ & $max(W(E),W(F))$ & $W(E)$
\\ \hline
\end{tabular}
\end{center}

Ako pri mnohých iných modeloch, aj do regulárnych výrazov vieme zakomponovať časti, ktoré nič nerobia (okrem toho, že zaberajú miesto). V rámci skúmania najjednoduchších výrazov sa prišlo k nasledujúcim definíciám \cite{newResults}:

\begin{df}
Nech E je regulárny výraz nad abecedou $\Sigma$ a nech L(E) je jazyk špecifikovaný výrazom E. Hovoríme, že E je \textbf{zmenšiteľný}, ak platí nejaká z nasledujúcich podmienok:
\begin{enumerate}
\item E obsahuje $\emptyset$ a $|E|>1$
\item E obsahuje podvýraz tvaru FG alebo GF, kde L(F)={$\varepsilon$}
\item E obsahuje podvýraz tvaru $F|G$ alebo $G|F$, kde $L(F)=\lbrace \varepsilon \rbrace$ a $\varepsilon \in L(G)$
\end{enumerate}
Inak, ak žiadna z nich neplatí, E nazývame \textbf{nezmenšiteľným}.
\end{df}

V originále sa používajú výrazy \textit{collapsible} a \textit{uncollapsible}. Táto definícia neodhalí všetky zbytočne zopakované časti, napríklad $a|a$ je nezmenšiteľný, aj keď by sa dal zapísať jednoduchým $a$. Problém je, že identity výrazov nie sú konečne axiomatizovateľné (ani nad unárnou abecedou)\cite{newResults}. Teda nie je reálne určiť také pravidlá, aby sme dosiahli konečné zjednodušenie. 

\begin{df}
Ak E je nezmenšiteľný regulárny výraz taký, že
\begin{enumerate}
\item E nemá nadbytočné (); a zároveň
\item E neobsahuje podvýraz tvaru $F**$
\end{enumerate}
potom vravíme, že E je \textbf{neredukovateľný} (irreducible).
\end{df}

Môžeme vyvodiť, že minimálny regulárny výraz pre daný jazyk bude nezmenšiteľný a neredukovateľný, naopak to však nemusí platiť. S takýmto základom možno dokázať napríklad tvrdenie: Ak E je neredukovateľný a $|alph(E)| \geq 1$, potom $|E| \leq 11|alph(E)| - 4$.

Iné spôsoby na ohraničenie regulárnych výrazov poskytujú nasledujúce pozorovanie a veta.

\begin{veta}[Proposition 6 \cite{newResults}]
Nech L je neprázdny regulárny jazyk.

(a) Ak dĺžka najkratšieho slova v L je $n$, potom $|alph(E)| \geq n$ pre ľubovoľný regulárny výraz E, kde L(E) = L.

(b) Ak naviac L je konečný a dĺžka najdlhšieho slova v L je $n$, potom $|alph(E)| \geq n$ pre ľubovoľný regulárny výraz, kde L(E) = L.
\end{veta}

Definícia non-returning NKA hovorí, že sa nevracia do počiatočného stavu, t.j. žiadne prechody nevedú smerom do $q_0$.

\begin{veta}[Theorem 10]
Nech E je regulárny jazyk s $|alph(E)| = n$. Potom existuje non-returning NKA akceptujúci $L(E)$ s $\leq n+1$ stavmi a DKA akceptujúci L(E) s $\leq 2n+1$ stavmi.
\end{veta}

Rozbehnutých je viacero oblastí skúmania. Regulárne výrazy sú zaujímavé hlavne preto, že tvoria stručnejší popis jazyka a sú ekvivalentné automatom. S tým súvisí prvá oblasť. Skúma sa stavová zložitosť automatu ekvivalentného konkrétnemu výrazu a naopak tiež popisná zložitosť výrazu ekvivalentného konkrétnemu automatu. Dá sa to zhrnúť ako problémy konverzií medzi modelmi. Niektorí autori siahajú po väčšej abstrakcii a namiesto automatov uvažujú iba orientované grafy s hranami označenými symbolmi. Určia si parametre grafu a snažia sa napríklad čo najlepšie popísať cesty medzi vrcholmi.

Ďalšia oblasť zahŕňa operácie nad regulárnymi výrazmi. Jeden z problémov je napríklad vzťah popisnej zložitosti výrazu pre jazyk $L$ a $L^c$. Pre automaty existuje konštrukcia pre vyrobenie komplementu jazyka. Avšak pre komplementárny regulárny výraz to nie je také jednoznačné.

Ako poslednú by sme spomenuli problém najkratšieho slova nešpecifikovaného re\-gu\-lár\-nym výrazom. Predpokladajme regulárny výraz $E$, kde $|alph(E)|=n$ nad konečnou abecedou $\Sigma$, pričom $L(E)\neq \Sigma^*$. Aké dlhé môže byť najkratšie slovo NEšpecifikované výrazom $E$? Najzaujímavejší výsledok je pre výraz s $|alph(E)| = 75n + 361$, kde naj\-krat\-šie nešpecifikované slovo je dĺžky $3(2 n - 1)(n + 1) + 3$.
